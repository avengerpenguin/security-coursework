\documentclass{report}

\usepackage[style=authoryear,backend=biber]{biblatex}

\title{The Folly of Passwords}
\author{Ross Fenning}

\bibliography{references}

\begin{document}

\maketitle
\tableofcontents

\chapter{Wireshark Analysis}

\chapter{GNU Privacy Guard (GPG)}

In this report, we look at how to set up GPG on a Debian or Ubuntu GNU/Linux
system although many of the steps apply to other Linux and Unix systems.

The steps outlined below will demonstrate how to:

\begin{enumerate}
\item generate a new private/public GPG key pair;
\item use the public key to encrypt a plain text message;
\item use the private key to descrypt the message; and
\item have a someone sign your new to verify that you own it.
\end{enumerate}

\section{Creating a GPG key pair}

The gpg tool is normally pre-installed in a lot of Linux/Unix systems. On
a Debian or Ubuntu GNU/Linux system that does not have it installed, it
can be installed quite simply with:

\begin{listing}
\$ sudo aptitude install gpg
\end{listing}

We can run the command to see there are no keys already generated:

\begin{listing}
  \$ gpg --list-keys
  gpg: directory `/home/ross/.gnupg' created
  gpg: new configuration file `/home/ross/.gnupg/gpg.conf' created
  gpg: WARNING: options in `/home/ross/.gnupg/gpg.conf' are not yet active during this run
  gpg: keyring `/home/ross/.gnupg/pubring.gpg' created
  gpg: /home/ross/.gnupg/trustdb.gpg: trustdb created
\end{listing}

Note that on the first use of the command, it creates some configuration files
and keyrings. If there are any keys already stored in the user's home
directory, then details of them would be printed out as well.

To create a key, we simply need to run:

\begin{listing}
$ gpg --gen-key
gpg (GnuPG) 1.4.12; Copyright (C) 2012 Free Software Foundation, Inc.
This is free software: you are free to change and redistribute it.
There is NO WARRANTY, to the extent permitted by law.

Please select what kind of key you want:
   (1) RSA and RSA (default)
   (2) DSA and Elgamal
   (3) DSA (sign only)
   (4) RSA (sign only)
Your selection? 1
RSA keys may be between 1024 and 4096 bits long.
What keysize do you want? (2048) 2048
Requested keysize is 2048 bits
Please specify how long the key should be valid.
         0 = key does not expire
      <n>  = key expires in n days
      <n>w = key expires in n weeks
      <n>m = key expires in n months
      <n>y = key expires in n years
Key is valid for? (0) 0
Key does not expire at all
Is this correct? (y/N) y

You need a user ID to identify your key; the software constructs the user ID
from the Real Name, Comment and Email Address in this form:
    "Heinrich Heine (Der Dichter) <heinrichh@duesseldorf.de>"

Real name: Ross Fenning
Email address: gpg@rossfenning.co.uk
Comment:
You selected this USER-ID:
    "Ross Fenning <gpg@rossfenning.co.uk>"

Change (N)ame, (C)omment, (E)mail or (O)kay/(Q)uit? O
You need a Passphrase to protect your secret key.

We need to generate a lot of random bytes. It is a good idea to perform
some other action (type on the keyboard, move the mouse, utilize the
disks) during the prime generation; this gives the random number
generator a better chance to gain enough entropy.
\end{listing}


\section{Encrypting a message}

\section{Decrypting the message}

\section{Signing the key}


\chapter{Ethical Hacking Lab Setup}

\end{document}
