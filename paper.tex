\documentclass{report}

\usepackage[style=authoryear,backend=biber]{biblatex}

\title{The Folly of Passwords}
\author{Ross Fenning}

\bibliography{references}

\begin{document}

\maketitle
\tableofcontents

\chapter{Introduction}

The use of passwords for authentication is certainly commonplace.
% TODO: How many passwords?
It is standard practice within small and large enterprises alike
to require employees to authenticate before using desktop
computers, mobile devices or centralised services (e.g.
as company email, corporate intranet or web-based self-service
systems for tasks such as booking annual leave).

These authentication requirements are largely driven by
several desires:

\begin{itemize}
  \item to prevent use of systems by non-employees;
  \item to maintain an audit trail of employees' usage;
  \item to protect company secrets and intellectual property;
  \item to safeguard the personal information of customers and employees; and
  \item to prevent economic and reputational damage from would-be attackers.
\end{itemize}

Additional security in any system normally comes with additional costs
in terms of usability for users and in financial terms for the organisation
that must implement any security processes and policies. However,
whilst many of the benefits and requirements listed above should justify
these costs, they are notably gains (or prevention of losses) for
the organisation itself more than they are gains for the individual
employees.

In chapter~\ref{chapter:costs}, we will explore the
cost-benefit trade-offs more closely and demonstrate that the status
quo unreaslistically transfers costs to employees to support gains
for the organisations.

Passwords are not the only approach to verifying a user's identity.
The methods by which a user can be authenticated are known to fall
into three categories -- or \emph{factors} -- of authentication
\parencite{council2005authentication}:

\begin{itemize}
  \item something the user \emph{knows}, such as a password, a PIN or
    personal information (e.g. mother's maiden name);
  \item something the user \emph{has}, such as a passport, a mobile device
    a paper ticket; and
  \item something the user \emph{is}, such as their DNA sequence,
    their retinal pattern, their fingerprint.
\end{itemize}

It should be noted that some methods can be argued as spanning
over two factors. For example, a credit card is something
a bank customer possesses, but the card number is something
an attacker could learn without the owner losing ownership
of the card. In the context of many online purchase, credit
card \emph{details} arguably become something a user \emph{knows}.

User authentication that requires two differing factors
of authentication (known as \emph{two-factor authentication}. There
has been a rise in two-factor authenication with companies
such as Google, Amazon, Microsoft, Paypal and Twitter having enabled
two-factor authentication for user accounts.

The mainstream adoption of authentication methods other than
passwords suggests increased ease in enterprises adoption means
of verifying users either instead of or in addition to
existing password setups.

In chapter~\ref{chapter:alternatives}, these alternatives will be
compared and assessed for their ease of adoption within a business.

\chapter{The Cost of Passwords}
\label{chapter:costs}

The username-password pair is arguably a \emph{de facto} standard
authentication mechanism employed by IT and security departments
within many businesses. The default install of any operating
system (Linux, Mac OS X, Windows, etc.) treats all user accounts
as consisting minimally of these pair of credentials and a
username-password login prompt is frequently the first
screen to greet users when a newly-installed OS is booted
for the first time.

In this chapter, we will explore the burden passwords place
on users, the challenges to usability they create and
the effectiveness of standard policies and procedures
frequently employed around password use.

%Linux and Mac systems make use of the Pluggable Authentication
%Mechanism (PAM), which weakens the assumption that authentication
%must be done with a password (password authenication is just one
%module available for PAM). Corporate systems typically make
%use of centralised user directories (e.g. Active Directory, OpenLDAP),
%which allows users to use common credentials across multiple
%IT systems (reducing the need for users to memorise multiple
%sets of credentials).

\section{Passwords and Usability}

\cite{adams1999users} famously asserted 15 years ago that security
departments take an inappropriate approach to security risk
by treating users as risk themselves to be controlled. They
argue further that security compromises are either knowingly
or unknowingly due to the way security is implemented more than
it is the direct fault of users themselves. Their recommendation
is that any security approaches need to be designed in
a user-centric way not just to increase usability, but to
reduce the likelihood of compromises as well.

Their work is still widely cited 15 years later, but
today, employees are increasingly using passwords in their
workplace in addition to a plethora of passwords for
other services used personally outside of work.
A user-centric design
approach to a modern, enterprise security setup needs to
be even more sympathetic to the large burden on users
from managing so many online accounts.

\cite{herley2009so} argues that users rationally reject costs
placed upon them from security policies. He invokes Kuhn's
``Paradigm Shift''\parencite{kuhn1962structure} to suggest
that the increasing burden placed on users with every attack
that takes place and the additional complexity that comes
therewith may hint at an imminent, yet dramatic change in
how users are asked to authenticate. It is yet to be seen
if Kuhn's observations of sudden shifts of thinking
in the scientific community apply to a more general
user base such as employees or indeed the general public, but
a strong need for true change can be taken from the analogy
at least.

%TODO: something about the large list of password rules

Additionally, SOMEONE explored economic arguments and models
to argue that the incredibly low, \emph{a priori} likelihood
of a user's account being compromised leads to a monetary
cost value far lower than that needed to justify a user
spending any real time at all strengthening their password.
This may be a case for a general, public account, but perhaps
some accounts with, e.g. corporate webmail, could be classed
as higher risk and bigger targets for attackers. An attack
on a corporate account might allow access to systems that
can lead to real economic or reputational damage.

We can argue that these costs -- even if higher than those
in SOMEONE's models -- are mainly suffered by the \emph{organisation}
more than the employees. This is very similar to SOMEONE's
arguments that even in the case of banks, they are likely
to take on the costs of fraud and prevent their customers
from exposure to those costs themselves.

This effect is classed
by SOMEONE as a negative externality on computer users in
that they are taking on additional effort to cover costs
that are not their own. These economic arguments can be
used to make a case for organisations taking more
responsibility and investing more to prevent the costs
from a security compromise. This can be combined with
the arguments for a more user-centric design
\parencite{adams1999users} to build a case for a new
authentication paradigm that is low friction to users and
where economic incentives lead to an equilibrium where
organisations spend exactly as much on security as
is rationally justified by their real risks and costs.

So, what is the nature of the burden experienced by users?
In the next few sections, we look at various aspects
of password security, how users are normally advised
and critically examine the effectiveness of that advice.

\section{Password Strength}
\label{sec:strength}

Password-protected accounts are vunerable to at least three
classes of attack\parencite{florencio2014password}:

\begin{itemize}
  \item Malware infection or other direct compromise of the client machine (e.g. keyloggers)
  \item Attacker learns the passwords via phishing, brute-force guessing or other means
  \item Access to a particular system is gained without the actual password (e.g session hijacking, cross-site request forgery)
\end{itemize}

The notion of \emph{password strength} describes how likely
a password is to suffer from an attacker learning the password via brute-force
guessing. A brute-force attack can be said to be \emph{online} (where the
attacker is sending multiple login attempts directly to a system protecting
by authentication) or \emph{offline} (where an attacker has a local copy of
an encrypted file or hashed password, to which she can perform an unlimited
number of attempts at cracking the password without arousing suspicion).

Information Theory and entropy can allow us to gauge a password's strength
in terms of \emph{bits}, which gives us an estimate of the likelihood of
an attacker guessing the password in one attempt. Passwords which are
more unlikely to guess have more bits and are thus consider stronger passwords.

The advice for creating strong passwords normally takes the form of tips
such as:

\begin{itemize}
  \item use longer passwords;
  \item avoid using real words;
  \item use a mix of lower and upper case; and
  \item include numbers and other non-alphanumeric characters.
\end{itemize}

\cite{kaufman2002network} facetiously claimed that humans are poor
at memorising strong passwords/keys:

\begin{quotation}
  \emph{
  Humans are incapable of securely storing high-quality
  cryptographic keys, and they have unacceptable speed and
  accuracy when performing cryptographic operations. (They are
  also large, expensive to maintain, difficult to manage, and they
  pollute the environment. It is astonishing that these devices
  continue to be manufactured and deployed. But they are
  sufficiently pervasive that we must design our protocols around
  their limitations.)
  }
\end{quotation}

This presents a problem if users are poor at working with the
very kind of passwords needed to avoid brute-force cracking. It
would suggest that a user-centric design would lead to something
more in line with abilities humans \emph{do} have.

% TODO: many = who?
Futhermore, many have noted that strong passwords defend
against \emph{only} that single attack vector. Learning the
password within guessing (e.g. via phishing) will put strong
passwords in the hands of attackers just as easily as with
weak passwords. Also, different classes of attacks entirely
such as installing keyloggers via malware or hijacking browser
sessions are unaffected by password strength.

It should also be noted that password strength is
used mainly to prevent
\emph{offline} brute force attacks as many systems
implement a policy of locking an account (or otherwise
delaying password attempts) when too many incorrect
password guesses are made. This effectively negates any
\emph{online} brute force attack against even a very weak
password.

%TODO: who?
Some have thus questioned whether the benefits from strong
passwords are as substantial as standard security advice
would suggest. However, this is perhaps still a concern
for an organisation as it likely does not want to suffer
a breach in an account that had a weak password. Some
online brute force attacks have been known to take place
where the attackers enumerate over all known user accounts
using a set of common or weak passwords. This increases
the probability that they gain access to at least one
account, which may be more useful than targeting a single
account.

So, if there is still an incentive for an organisation
to ensure its employees' accounts are secure, but there
are few pressures (or even rational justification)
for the individual users to spend their time and effort doing
so, what can the organisation do?

One classical solution to weak passwords
has been to enforce password strength \emph{requirements}.
This is a technical solution whereby the systems simply
reject self-selected user passwords that do not conform
to a target strength or entropy.
This is a technical solution that shifts the burden forcibly onto
users and we can argue it is susceptible to the deliberate workarounds
that many users ultimately employ to reduce the overheads
in performing their primary jobs.

Another, softer approach is to attempt to educate users
on choosing strong, yet memorable, passwords. This belief
that weak passwords are always down to lack of training
is fundementally challenged in the influential findings
of \cite{adams1999users}. Their work combined with the
economics-based arguments and models introduced by %WHO?
builds a large weight of evidence that well-informed
users will still not follow much security advice
(sometimes rationally so) to reduce their own costs in
time and effort or to remove a significant barrier
to performing their work tasks.

\section{Password Expiry}

Brute-force password guessing can also be mitigated by
users changing passwords frequently enough that attackers
have a short window within which to guess a particular
password.

As with password strength, this is only mitigating the
brute-force guessing attack vector as phishing, keylogging, etc.
can still give an attacker a password for a short while -- until
the password is next changed. As with password strength, this
is also most effective against offline brute-force attacks (e.g.
the attacker has acquired a list of hashed passwords) an
account locking policy also prevents an online brute-force
attack without the need for passwords to expire.

It is a fairly common practice in many organisations to
implement a password expiration policy whereby the
authentication systems will require a user to change their
password after some defined period of time.

\cite{adams1999users} note that password expiry increases
the burden on users having to memorise multiple passwords and that
50\% of users in their study wrote passwords down, with one
person citing a monthly password expiration policy as the cause
of having to do so.

\cite{zhang2010security} go further to demonstrate the ease of
algorithmically predicting a user's new password when the old,
expired password is known. They assert that password expiration
is weak at achieving the intent behind its implementation and
support the idea of moving away from passwords entirely.

\section{Password Reuse}

Security advice given to users usually includes the suggestion that users
have a unique password for each service they use. This can
be an unreasonable burden on users\parencite{florencio2014password}
and is thus an unrealistic expectation for an organisation or online
service.

\cite{preibusch2010password} showed how the pressure to reuse passwords
presents negative externalities from security-indifferent websites upon
security-concerned websites. This presents an organisation with
security risks stemming solely from the existence of other, less
secure websites with which its employees have accounts.

The secret nature of passwords means an organisation cannot \emph{enforce}
the use of unique passwords, so it must either make reuse impossible or
reduce the impact from reuse. Businesses can make reuse impossible
by having password strength requirements that are stronger than
services less concerned with security, with which employeess may also have
accounts. This is susceptible to the issues in section~\ref{sec:strength}
where this increases costs on users over and above than those they will
suffer from a security breach.

Clearly, password reuse is a coping strategy for users arising from the
difficulty humans have with passwords as an authentication mechanism. This
adds weight further to the case for a real shift away from passwords in
general. We will explore this case in section~\ref{sec:case} and evaluate
the alternatives to password use in chapter~\ref{chapter:alternatives}.

\section{The Case for a Paradigm Shift}

In this chapter, we have built up the following claims:

\begin{enumerate}
  \item Strong passwords reduce guessability, but do not protect against
    many other attacks whilst adding an economically inefficient cost to users'
    time and effort.
  \item Password expiration policies are similar in only protecting against
    brute-force guessing and their effectiveness is called into question when
    users use simple mutations to choose their next password.
  \item Organisations and websites do not exist in isolation, but instead in
    an ecosystem where their users have passwords on other services and are likely
    to reuse them to reduce the burden of memorising too many.
\end{enumerate}

In summary, the use of passwords is problematic for organisations and policies
to reduce those problems place the costs on the wrong people: the users. We can
make the case here for a shift away from passwords in general or at least
for further research into user-centric designs that handle the costs of security
rationally with suitable consideration given to usability.

In chapter~\ref{chapter:alternatives}, we look at some alternatives to passwords
and evaluate them.

\chapter{Comparative Analysis of Password Alternatives}
\label{chapter:alternatives}

\chapter{Conclusions and Recommendations}

\chapter{Further Work}

\printbibliography

\end{document}
