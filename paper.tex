\documentclass{report}

\usepackage[style=authoryear,backend=biber]{biblatex}

\title{The Folly of Passwords}
\author{Ross Fenning}

\bibliography{references}

\begin{document}

\maketitle
\tableofcontents

\chapter{Introduction}

The use of passwords for authentication is certainly commonplace.
% TODO: How many passwords?
It is standard practice within small and large enterprises alike
to require employees to authenticate before using desktop
computers, mobile devices or centralised services (e.g.
as company email, corporate intranet or web-based self-service
systems for tasks such as booking annual leave).

These authentication requirements are largely driven by
several desires:

\begin{itemize}
  \item to prevent use of systems by non-employees;
  \item to maintain an audit trail of employees' usage;
  \item to protect company secrets and intellectual property;
  \item to safeguard the personal information of customers and employees; and
  \item to prevent economic and reputational damage from would-be attackers.
\end{itemize}

Additional security in any system normally comes with additional costs
in terms of usability for users and in financial terms for the organisation
that must implement any security processes and policies. However,
whilst many of the benefits and requirements listed above should justify
these costs, they are notably gains (or prevention of losses) for
the organisation itself more than they are gains for the individual
employees.

In chapter~\ref{chapter:costs}, we will explore the
cost-benefit trade-offs more closely and demonstrate that the status
quo unreaslistically transfers costs to employees to support gains
for the organisations.

Passwords are not the only approach to verifying a user's identity.
The methods by which a user can be authenticated are known to fall
into three categories -- or \emph{factors} -- of authentication
\parencite{council2005authentication}:

\begin{itemize}
  \item something the user \emph{knows}, such as a password, a PIN or
    personal information (e.g. mother's maiden name);
  \item something the user \emph{has}, such as a passport, a mobile device
    a paper ticket; and
  \item something the user \emph{is}, such as their DNA sequence,
    their retinal pattern, their fingerprint.
\end{itemize}

It should be noted that some methods can be argued as spanning
over two factors. For example, a credit card is something
a bank customer possesses, but the card number is something
an attacker could learn without the owner losing ownership
of the card. In the context of many online purchase, credit
card \emph{details} arguably become something a user \emph{knows}.

User authentication that requires two differing factors
of authentication (known as \emph{two-factor authentication}. There
has been a rise in two-factor authenication with companies
such as Google, Amazon, Microsoft, Paypal and Twitter having enabled
two-factor authentication for user accounts.

The mainstream adoption of authentication methods other than
passwords suggests increased ease in enterprises adoption means
of verifying users either instead of or in addition to
existing password setups.

In chapter~\ref{chapter:alternatives}, these alternatives will be
compared and assessed for their ease of adoption within a business.

\chapter{Problems with Passwords}
\label{chapter:costs}

\section{The Costs of Passwords}
\section{Password Strength}
\section{Expiry}

\chapter{Comparative Analysis of Password Alternatives}
\label{chapter:alternatives}

\chapter{Conclusions and Recommendations}

\chapter{Further Work}

\printbibliography

\end{document}
